% !TEX TS-program = xelatex
% !TEX encoding = UTF-8 Unicode

\documentclass{scrarticle}

% Language setting
% Replace `english' with e.g. `spanish' to change the document language
\usepackage[english]{babel}

% Set page size and margins
% Replace `letterpaper' with`a4paper' for UK/EU standard size
\usepackage[a4paper ,top=2cm,bottom=2cm,left=3cm,right=3cm]{geometry}

% Useful packages
\usepackage{amsmath}
\usepackage{graphicx}
\usepackage[colorlinks=true, allcolors=blue]{hyperref}

% Bibliography
%\usepackage[backend=biber]{biblatex} %Imports biblatex package
%\addbibresource{sample.bib} %Import the bibliography file

% Paragraph formatting
\setlength{\parindent}{0em}
\setlength{\parskip}{2ex}


\title{Template for PhD Project Description, Department of Informatics, UiO}
\author{Author name}

\begin{document}
\maketitle

\textbf{This template was approved by the PhD committee at the Department of Informatics, and PhD project proposal should adhere to the template, effective from March 1st 2020.}

The PhD thesis is an independent, scientific work that meets international standards regarding academic level, method, and ethical requirements. The preferred format of the thesis is a collection of publication/articles published in internationally recognized publication channels (scientific journals or conferences).

The PhD degree at the Department of Informatics is a supervised study requiring a project description for admission to the PhD program. The project description is a joint proposal to be approved and signed by the PhD candidate and the supervisors. The main purpose of the project description is to ensure a smooth collaboration and good progress with proper research focus and clear milestones. The project description should be detailed enough for the PhD Committee at IFI and the Faculty to assess research goals, scientific challenges, incremental milestones and publication plan. It also forms the baseline for the third semester evaluation.

\textbf{The project description should be between 4 – 10 pages, and contain the following nine elements:}

\section{Project title}

\section{Main objective and summary of the project}

Present the main objective and a brief summary, explaining how you intend to attain your goal.

\section{Project background and scientific basis}

Provide a brief survey of existing research efforts and scientific basis, preferably with references to scientific literature. If your project is a part of a larger research effort requiring coordination, explain specifically your planned contributions within the larger framework.

\section{Research questions and scientific challenges}

Outline your research questions and scientific challenges. Explain their scientific foundation and your approach to address them. Describe your ambitions of reaching beyond the state-of- the-art and how to reach these goals.

\section{Scientific method}

Outline your method for reaching your scientific goals. Describe how you are planning to use empirical, analytical or other methods for your research. Please state how these methods relate to the expected scientific contributions of the thesis. If applicable, describe the source of your data.

\section{Expected impact}

Describe the expected impact of the research on science, innovation and society.

\section{Ethics}

Describe any research-ethical challenges related to your project. Use the applicable research- ethical guidelines of the Norwegian Research Ethical Committees\footnote{The guidelines of the National Committee for Research Ethics in Science and Technology are central here. Also the guidelines of NESH and the guidelines on Internet Research may be applicable. See the Norwegian Research Ethical Committees: https://www.etikkom.no/en/} to identify such challenges and propose means to address these. Also, check if your project requires notification to the Data Protection Official for Research\footnote{Data Protection Official for Research: http://www.nsd.uib.no/nsd/english/pvo.html}. Candidates doing fieldwork abroad should check if they also have to register their project with the applicable authorities there. All PhD candidates registered at IFI, independent of their nationality or the country where they do fieldwork, need clearance from the Norwegian Data Authority represented by the NSD (http://www.nsd.uib.no) if they are going to process personal data. NSD provides a simple tool to assess if a notification is necessary as well as the notification form (https://nsd.no/personvernombud/en/notify/index.html).

\section{Project timeline}

Outline a research plan for each semester including course-work, scientific contributions, and planned publications. Preferably, indicate publication channel as well. Indicate mandatory/teaching duties (25\%) when applicable as well as planned visits to other research institutions (national or international). We use the project timeline to assess progress during the third semester evaluation.


We recommend you to summarize the research plan in a table to make it easy to evaluate progress. This can for example look like:

\begin{table}[h]
	\begin{tabular}{ll}
		\hline
		\textbf{Semester} & \textbf{Activities and milestones} \\ \hline
		1st & \begin{tabular}[c]{@{}l@{}}Research Activities\\ Courses\\ Conference attendance\\ Submission of publications (theme, publication channel)\\ Research stays abroad\\ Mandatory duties\\ Etc.\end{tabular} \\ \hline
		2nd & ... \\ \hline
		3rd & \begin{tabular}[c]{@{}l@{}}3rd semester reporting\\ ...\end{tabular} \\ \hline
		4th & ... \\ \hline
		5th & ... \\ \hline
		... & ... \\ \hline
	\end{tabular}
\end{table}

\section{Project organisation and cooperation}

The principal supervisor should provide an overview of knowledge / expertise each supervisor will contribute, how they collectively cover the academic field of the project, and how the members of the supervisory team will cooperate. One should also describe how the support of the candidate will be maintained if one of the members has to vacate her/his post and therefore has to be replaced. Finally, the robustness of the research group should be addressed; i.e., how other academic staff and their PhD-students and PhD projects support the new project and how the new project fits into the ongoing research activity.

\section{Literature references}

List applicable references.


%\printbibliography %Prints bibliography

\end{document}
